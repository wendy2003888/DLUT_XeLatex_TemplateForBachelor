% !TEX TS-program = XeLaTeX
% !TEX encoding = UTF-8 Unicode

\BiChapter{结论与展望}{The Summarize and Prospect}


该部分主要包括两部分:“结论”和“展望”。结论是理论分析和实验结果的逻辑发展,是整篇论文的归宿。结论是在理论分析、试验结果的基础上,经过分析、推理、判断、归纳的过程而形成的总观点。结论必须完整、准确、鲜明、并突出与前人不同的新见解。总结部分还应说明论文中的创新点内容。创新点应该以分条列举的形式进行提出。展望是对该研究课题存在的不足和有待改进的说明,是对未来研究的一种期待。该部分的字数应不少于半页。

书写格式说明:
标题“结论与展望”选用模板中的样式所定义的“结论与展望”,或者手动设置成字体:黑体,居中,字号:小三,1.5倍行距,段后1行,段前为0行。
结论正文选用模板中的样式所定义的“正文”,每段落首行缩进2字;或者手动设置成每段落首行缩进2字,字体:宋体,字号:小四,行距:多倍行距 1.25,间距:前段、后段均为0行。
