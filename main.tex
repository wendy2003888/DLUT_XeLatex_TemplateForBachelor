% !TEX TS-program = XeLaTeX
% !TEX encoding = UTF-8 Unicode

%%%%%%%%%%%%%%%%%%%%%%%%%%%%%%%%%%%%%%%%%%%%%%%%%%%%%%%%%%%%%%%%%%%%%%
%
%	大连理工大学本科论文 XeLaTeX 模版 —— 主文件 main.tex
% 版本:0.86
% 最后更新:2014.05.22
% 修改者:wendy2003888 
% 修订者:zzupc(E-mail: zzupchd@gmail.com;微博:http://weibo.com/zzupc)
% 编译环境1:Ubuntu 12.04 + TeXLive 2011 + Emacs
% 编译环境2:Windows 7 + CTeX v2.9.2.164 + WinEdit
%编译环境3:Mac OS + Mac LaTex + Texshop
%%%%%%%%%%%%%%%%%%%%%%%%%%%%%%%%%%%%%%%%%%%%%%%%%%%%%%%%%%%%%%%%%%%%%%

\documentclass[12pt, a4paper, openany, oneside, x11names]{book}
\usepackage{pdflscape}

% 字体配置文件
\input{setup/fonts}


% 宏包配置文件
\input{setup/packages}

% 格式文件
\input{setup/format}




\begin{document}
%代码块设置
\lstset{
backgroundcolor=\color{white},
basicstyle=\Roman\xiaosi, 
numberstyle=\Roman\xiaosi
}
% 定义所有的图片文件在 figures 子目录下
\graphicspath{{figures/}}

% 前言
\frontmatter
\pagenumbering{Roman}
\input{preface/cover}       	% 封面
%\includepdf{preface/cover1.pdf}
%\cleardoublepage
%\includepdf{preface/cover2.pdf}
%\cleardoublepage
\cleardoublepage
\makeabstract

%\setcounter{tocdepth}{3}

% 设置目录字体和行间距

\defaultmenufont
% 目录
\tableofcontents

\cleardoublepage

\defaultfont
\mainmatter
% 正文章节
\defaultfont
\renewcommand{\thefootnote}{\arabic{footnote}}

\include{body/chap00}
\include{body/chap01}
\include{body/chap02}
% !TEX TS-program = XeLaTeX
% !TEX encoding = UTF-8 Unicode

\BiChapter{注意事项}{Notes}
%\label{chap03}
\BiSection{引~~言}{Introduction}
请直接双面打印PDF文件,空白页已经按要求留出。打印时,缩放页面的选项设
为“无”,否则页面会缩小。

参考文献的bib文件的条目的名称是不允许出现空格的。

% 结论
% !TEX TS-program = XeLaTeX
% !TEX encoding = UTF-8 Unicode

\BiChapter{结论与展望}{The Summarize and Prospect}


该部分主要包括两部分:“结论”和“展望”。结论是理论分析和实验结果的逻辑发展,是整篇论文的归宿。结论是在理论分析、试验结果的基础上,经过分析、推理、判断、归纳的过程而形成的总观点。结论必须完整、准确、鲜明、并突出与前人不同的新见解。总结部分还应说明论文中的创新点内容。创新点应该以分条列举的形式进行提出。展望是对该研究课题存在的不足和有待改进的说明,是对未来研究的一种期待。该部分的字数应不少于半页。

书写格式说明:
标题“结论与展望”选用模板中的样式所定义的“结论与展望”,或者手动设置成字体:黑体,居中,字号:小三,1.5倍行距,段后1行,段前为0行。
结论正文选用模板中的样式所定义的“正文”,每段落首行缩进2字;或者手动设置成每段落首行缩进2字,字体:宋体,字号:小四,行距:多倍行距 1.25,间距:前段、后段均为0行。

\cleardoublepage
\backmatter

% 参考文献
\addcontentsline{toc}{chapter}{参考文献}
\addcontentsline{toe}{chapter}{References} % 参考文献加入到英文目录 by zzupc
\defaultfont
\wuhao
\bibliographystyle{zjugbno}
\bibliography{body/reference}
\cleardoublepage
% 附录
\defaultfont
\begin{appendix}
  \input{appendix/chapA}
\end{appendix}
\cleardoublepage

\defaultfont
% 致谢
\include{appendix/acknowledgements}
\cleardoublepage


\end{document}
