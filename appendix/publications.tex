% !TEX TS-program = XeLaTeX
% !TEX encoding = UTF-8 Unicode

%%%%%%%%%%%%%%%%%%%%%%%%%%%%%%%%%%%%%%%%%%%%%%%%%%%%%%%%%%%%%%%%%%%%%
%
%	大连理工大学博士论文 XeLaTeX 模版 —— 发表论文文件 publications.tex
% 版本:0.8
% 最后更新:2012.04.04
% 修改者:Yuri (E-mail: yuri_1985@163.com)
% 修订者:whufanwei(E-mail: dutfanwei@qq.com)
% 编译环境1:Ubuntu 12.04 + TeXLive 2011 + Emacs
% 编译环境2:Windows 7 + CTeX v2.9.2.164 + WinEdit
%
%%%%%%%%%%%%%%%%%%%%%%%%%%%%%%%%%%%%%%%%%%%%%%%%%%%%%%%%%%%%%%%%%%%%%

\chapter*{\hfill 攻读博士学位期间科研项目及科研成果 \hfill}
\addcontentsline{toc}{chapter}{攻读博士学位期间科研项目及科研成果}
\addcontentsline{toe}{chapter}{Achievements}
%仅列出博士生攻读博士学位期间发表与学位论文有关的学术论文,
%并注明属于学位论文内容的部分(章节),
%所有作者及其顺序、所发表的刊物名称(包括主办单位、是否被SCI、EI检索期刊)、时间、期号与页码。
%其他时间或与学位论文内容(章节)无关的论文不得列出。示例如下:
首先列出博士生攻读博士学位期间发表与学位论文有关的学术论文,并注明属于学位论文内容的部分(章节),所有作者及其顺序、所发表的刊物名称(包括主办单位、是否被SCI、EI检索期刊)、时间、期号与页码。其他时间或与学位论文内容(章节)无关的论文不得列出。特别声明:对于已经投稿但是没有确认被接受的论文不得写入论文中,列出的论文只能是已经确定发表或者已经确认接收的论文。

其次列出在攻读博士学位期间参与的科研项目(如国家自然科学基金或国家“863”计划等),以及在这期间取得的科研成果(申请的发明专利、在国家级赛事中取得的优异成绩等)。

书写格式说明:
标题“攻读博士学位期间科研项目及科研成果”选用模板中的样式所定义的“发表学术论文情况”;或者手动设置成字体:黑体,居中,字号:小三,1.5倍行距,段后1行,段前为0行。

“攻读博士学位期间科研项目及科研成果”正文选用模板中的样式所定义的“正文”,每段落首行缩进2字;或者手动设置成每段落首行缩进2字,字体:宋体,字号:小四,行距:多倍行距 1.25,间距:前段、后段均为0行。


%\renewcommand{\labelenumi}{[\arabic{enumi}]}



\section*{\underline{攻读博士学位期间发表的学术论文}}
\begin{publist}
\item L. Wang, S. Kang, H. Shum, G. Xu, Error Analysis of Pure
  Rotation-based Self-Calibration, {\em{IEEE Transactions on Pattern
      Analysis and Machine Intelligence (PAMI)}}, in press
\item ×××,××,×××. 一种基于全景图的三维房间导航方法.
  软件学报, 2002, 13(Suppl.): 31-35
\end{publist}


%\section*{在国际和国内学术刊物上发表的论文}
%\begin{enumerate}[label={[\arabic*]}]
%\item L. Wang, S. Kang, H. Shum, G. Xu, Error Analysis of Pure
%  Rotation-based Self-Calibration, {\em{IEEE Transactions on Pattern
%      Analysis and Machine Intelligence (PAMI)}}, in press
%\item ×××,××,×××. 一种基于全景图的三维房间导航方法.
%  软件学报, 2002, 13(Suppl.): 31-35
%\end{enumerate}
%
%\section*{在国际和国内学术会议上发表的论文}
%\begin{enumerate}[label={[\arabic*]}]
%\item L. Wang, S. Kang, H. Shum, G. Xu, Error Analysis of Pure
%  Rotation-based Self-Calibration, {\em{in Proceedings of the Eighth
%      IEEE International Conference on Computer Vision(ICCV'01)}}, I:
%  464-471, Vancouver, BC, Canada, July, 2001
%\item L. Wang, X. Liu, L. Xia, G. Xu, A. Bruckstein, Image
%  Orientation Detection with Integrated Human Perception Cues,
%  {\em{in Proceedings of IEEE International Conference on Image
%      Processing (ICIP'03)}}, in press
%\end{enumerate}



